\documentclass[12pt, a4paper]{article}

\usepackage[utf8]{inputenc}
\usepackage[T1]{fontenc}
\usepackage{amsmath}
\usepackage{amssymb}
\usepackage{geometry}
\geometry{a4paper, margin=1in}
\usepackage{fancyhdr}

\pagestyle{fancy}

\fancyhf{}
\rhead{Übungsblatt 2}
\lhead{Nick Martin, Tim Lukas}
\cfoot{\thepage}
\renewcommand{\headrulewidth}{0.4pt}
\renewcommand{\footrulewidth}{0.4pt}

\begin{document}

\begin{center}
    \vspace*{1cm}
    {\Large \textbf{Übungsblatt 2 - Informationsvisualisierung und Visual Analytics}}
\end{center}
\begin{center}
    {\Large Nick Martin, Tim Lukas}
    \vspace*{1cm}
\end{center}

\section*{Aufgabe 1: Daten- und Task-Analyse}

\subsection*{\textit{What?} (Datenanalyse)}
Bei den gegebenen Rohdaten handelt es sich um eine zweidimensionale Tabelle.
Die Items (Zeilen dieser Tabelle) sind sechs individuelle Personen identifiziert von A bis F, wobei jede Person eine eindeutige Entität darstellt.
Diese Personen werden durch fünf Attribute (Spalten) beschrieben: dabei handelt es sich um die Programmiersprachen Java, Perl, JavaScript, C\#, Python.
In den Zellen der Tabelle werden die jeweiligen Werte angegeben, welche die Selbsteinschätzung der jeweiligen Person für eine spezifische Programmiersprache
repräsentieren. Diese Werte liegen auf einer Skala von 1 bis 5, bei diesen Werten gilt: Je höher der Wert, desto besser die Selbsteinschätzung in der jeweiligen Programmiersprache.

% Aufzählung durch darüberliegenden Fließtext ersetzt
% \begin{itemize}
%   \item \textbf{Datentyp:} bei den Rohdaten handelt es sich um eine \textbf{Tabelle}.
%   \item \textbf{Items (Zeilen):} Die Items sind die 6 \textbf{Personen} (A - F). Jede Person stellt eine Entität dar. 
%   \item \textbf{Attribute (Spalte):} Die Attribute sind die 5 \textbf{Programmiersprachen} (Java, Perl, JS, C\#, Python). 
%     Sie beschreiben Eigenschaften der Items. 
%   \item \textbf{Werte (Zellen):} Die Werte in den Zellen repräsentieren die Selbsteinschätzung der Kenntnisse einer Person für eine 
%     bestimmte Programmiersprache, die Werte liegen auf einer Skala von 1-5. 
% \end{itemize}

\paragraph{Attirbut Typen}
die Attribute der Tabelle entsprechen folgenden Klassifikationen:
\begin{itemize}
  \item \textbf{Personen:} Personen sind kategorische Attribute. Sie haben keine implizite Ordnung und dienen in dieser Tabelle 
    als eindeutiger Schlüssel, um auf die Werte eines Items zuzugreifen.
  \item \textbf{Programmiersprachen:} Auch die Programmiersprachen sind kategorische Attribute.
  \item \textbf{Werte der Selbsteinschätzung:} Die Selbsteinschätzungen sind ordinale Attribute. Es existiert eine klare Rangfolge (5 ist besser als 4),
    die Abstände zwischen den Werten sind jedoch weder quantifizierbar noch notwendigerweise gleichmäßig.
\end{itemize}



\subsection*{Why? (Task)}
Bei der gegeben Fragestellung \emph{"Wer ist der beste und wer der schlechteste Programmierer?"} handelt es sich um eine Domänenfrage.\newline
Die Frage ist:
\begin{itemize}
  \item \textbf{Mehrdeutig:} Die Definition von "bester" ist nicht klar. Ist es die Person mit konstant hohen Werten oder eher die Person mit Spitzenwerten
    in wenigen Kategorien?
  \item \textbf{Unpräzise:} Es wird keine klare Metrik für den Vergleich vorgegeben.
  \item \textbf{Nicht direkt ausführbar:} Die Frage kann nicht ohne weitere Annahmen direkt an die Daten gestellt werden.
\end{itemize}

Die Frage könnte beispielsweise in folgende Datenfragen übersetzt werden:
\begin{center}
  \textit{"Welche Person hat die höchste Summe an Fähigkeitspunkten über alle fünf Programmiersprachen und welche die niedrigste?"}
\end{center}


\paragraph{Abstraktion der Aufgabe:}
Das übergeordnete Ziel ist der Vergleich der Programmierfähigkeiten aller Personen, um die Extremwerte zu erkennen.
Um dieses Ziel zu erreichen, sind folgende Aktionen notwendig:
\begin{enumerate}
  \item \textbf{Zusammenfassen / Agregieren:}
    Da jede Person durch mehrere Attribute (fünf Programmiersprachen) beschrieben wird, kann kein direkter Verlgeich stattfinden. Zuerst müssen diese verschiedenen Werte
    für jede Person zu einem einzigen, repräsentativen Gesamtwert zusammengefasst werden.
      \begin{itemize}
        \item \textbf{Beispiel:} Berechnung der Summe oder des Medians der Fähigkeitswerte pro Person.
      \end{itemize}
  \item \textbf{Ordnen:} Die aus der Aggregation resultierenden Gesamtwerte können dann verwendet werden, um eine Rangliste aller Personen zu erstellen. Diese Aktion bringt
    die Items(Personen) in eine explizite Reihenfolge.
  \item \textbf{Suchen/Identifizieren:} Im finalen Schritt müssen die beiden gesuchten Items aus der im vorherigen Schritt erstellten, geordneten Liste extrahiert werden:
    Die erste Person (der "beste") und die letzte Person (der "schlechteste").
\end{enumerate}


\vspace{1em}
\textbf{Ergebnis der Task-Abstraktion:} Die Aufgabe besteht darin, die Attribute einer Menge von Items (Personen) zu aggregieren,
um eine Rangordnung zu erstellen, mit der die Extremwerte identifiziert werden können.



\section*{Aufgabe 2 - Datenattribute}
\subsection*{Diskussion zwischen den in der Vorlesung vorgestellten und in der Publikation vorgeschlagenen Attribut-Typen}
Der primäre Unterschied zwischen der Klassifikation von S. S. Stevens und dem in der Vorlesung behandelten System liegt in ihrer jeweiligen Ausrichtung.
Stevens' Gliederung in Nominal-, Ordinal-, Intervall- und Verhältnisskalen ist auf die statistische Analyse ausgelegt.
Sie definiert, welche mathematischen Operationen für einen Datentyp zulässig sind, um die statistische Validität gewährleisten zu können.
Die in der Vorlesung behandelten Kategorien Kategorial, Geordnet und Quantitativ sind hingegen auf die grafische Umsetzung ausgelegt. 
Hierbei ist die zentrale Frage, wie die Struktur der Daten am effektivsten und sinngemäß durch visuelle Kanäle (z.B. Farbe, Form und Position) dargestellt werden kann.

Betrachtet man die jeweils vorgeschlagenen Attributstypen fällt auf, dass Stevens' Nominalskala direkt dem in der Vorlesung behandelten kategorialen Attributstyp
entspricht. Die Ordinalskala entspricht dem geordneten (ordered) Typ.

Die größte Abweichung betrifft die numerischen Daten. Stevens unterscheidet klar zwischen Intervallskalen, die keinen echten Nullpunkt besitzen (z.B. Celsius), und
Verhältnisskalen, die einen absoluten Nullpunkt aufweisen (z.B. Körpergröße).
Diese Unterscheidung ist statistisch relevant, da Verhältnisse nur dann sinnvoll gebildet werden können, wenn ein absoluter Nullpunkt existiert.
Im in der Vorlesung behandelten Modell werden beide Skalentypen zu einem gemeinsamen quantitativen Typ zusammengefasst.
Der Grund dafür ist, dass bei der grafischen Darstellung die Art des Nullpunkts in der Regel keine Rolle spielt.
Sowohl Intervall- als auch Verhältnisskalen werden üblicherweise über dieselben visuellen Merkmale dargestellt, etwa durch die Position auf einer Achse oder die Länge eines Balkens.

Abschließend lässt sich sagen, dass das in der Vorlesung behandelte Modell eine für den Anwendungsfall der Informationsvisualisierung optimierte Abstraktion ist. 
Es vereinfacht Stevens' strenge statistische Regeln zu einem Modell, das Design-Entscheidungen für die effektive visuelle Repräsentation von Daten anleitet.


\end{document}
