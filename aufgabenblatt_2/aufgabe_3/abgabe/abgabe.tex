\documentclass[a4paper,12pt]{article}

% --- Pakete ---
\usepackage[utf8]{inputenc}
\usepackage[T1]{fontenc}
\usepackage[ngerman]{babel}
\usepackage{graphicx}
\usepackage{geometry}
\geometry{margin=2.5cm}

% --- Titelinformationen ---
\title{Aufgabenblatt 2 (Aufgabe 3): Kurzbeschreibung zur Datenvisualisierung}
\author{Nick Martin, Tim Lukas}
\date{\today}

\begin{document}

\maketitle

\section*{Erklärung zur Visualisierung}

Die Visualisierung zeigt den Zusammenhang zwischen dem durchschnittlichen Tageseinkommen (x-Achse)
und der durchschnittlichen Kinderzahl pro Frau (y-Achse) in verschiedenen Ländern.
Die Größe der Kreise repräsentiert die jeweilige Bevölkerungszahl, während die Farbe die Zugehörigkeit zu einem Kontinent anzeigt.

Es ist ein deutlicher negativer Zusammenhang erkennbar: Mit steigendem Einkommen sinkt die durchschnittliche
Kinderzahl pro Frau. Länder mit niedrigerem Einkommen weisen tendenziell höhere Geburtenraten auf.
Dieser Zusammenhang wird besonders im Vergleich zwischen den Kontinenten Europa und Afrika deutlich. Während der Großteil der
afrikanischen Länder eine hohe Geburtenrate bei geringem Durchschnittseinkommen aufweist, zeigt sich in den eruopäischen Ländern
ein gegenteiliges Bild: Die meisten verfügen über höhere Einkommen und niedrigere Geburtenraten.

Die Visualisierung eignet sich folglich, um die Zusammenhänge zwischen der ökonomischen Situation eines Lands bzw. dessen Einwohnern
und dessen demografischer Entwicklung aufzuzeigen. Zudem ermöglicht sie einen direkten Vergleich zwischen den einzelnen Ländern und Kontinenten.

\end{document}
